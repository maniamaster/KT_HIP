\NeedsTeXFormat{LaTeX2e}[2005/12/01]
%%    2010/04/06 v1.0 Vorlage Master-Forschungspraktikum Versuchsauswertung
%%    based on the 2009/10/14 v0.1 GAUBM template by Prof Pruschke

\documentclass[twoside,        %% zweiseitiges Layout
               BCOR12mm,       %% Bindekorrektur 12 mm
% please comment out if report is in English
%               english,ngerman, %% Dokumentspr. Deutsch, Alternativspr. Englisch
% please remove comment if report is in English 
               ngerman,english, %% Dokumentspr. Englisch, Alternativspr. Deutsch
               fleqn,headsepline=false,footsepline=false
              ]{Vorlage/MFPREPORT}
\makeatletter
\DeclareOldFontCommand{\rm}{\normalfont\rmfamily}{\mathrm}
\DeclareOldFontCommand{\sf}{\normalfont\sffamily}{\mathsf}
\DeclareOldFontCommand{\tt}{\normalfont\ttfamily}{\mathtt}
\DeclareOldFontCommand{\bf}{\normalfont\bfseries}{\mathbf}
\DeclareOldFontCommand{\it}{\normalfont\itshape}{\mathit}
\DeclareOldFontCommand{\sl}{\normalfont\slshape}{\@nomath\sl}
\DeclareOldFontCommand{\sc}{\normalfont\scshape}{\@nomath\sc}
\makeatother

%% Pakete und Definitionen ausgelagert
\input{Vorlage/packages}
\input{Vorlage/layout}

%% einbinden einiger nützlicher Befehle
\input{Vorlage/shortcuts}

%Zur Formatierung in der Matheumgebung
\renewcommand{\t}{\ensuremath{\rm\tiny}} % Tiefgestellter Text in der Matheumgebung wird schoener mit: $\Phi_{\t{Text}}$
\renewcommand{\d}{\ensuremath{\mathrm{d}}} % Die totale Ableitung ist stets aufrecht zu setzen: \d
\newcommand{\diff}[3][]{\ensuremath{\frac{\d^{#1}#2}{\d#3^{#1}}}} % einfache Ableitung nach x: $\ddx{\Phi}$
\newcommand{\pdiff}[3][]{\ensuremath{\frac{\partial^{#1}#2}{\partial#3^{#1}}}} % wie gesprochen, eine partielle Ableitung: \del
\newcommand{\aeqiv}{\ensuremath{\qquad \Longleftrightarrow \qquad}} % Eine Aequivalenz
\newcommand{\folgt}{\ensuremath{\qquad \Longrightarrow \qquad}} % Ein Folgepfeil mit Abstaenden
\newcommand{\corresponds}{\ensuremath{\mathrel{\widehat{=}}}} % Befehl für "Entspricht"-Zeichen
\newcommand{\mi}[1]{\ensuremath{\mathit{#1}}} % italics für griechische Buchstaben in Matheumgebung

%Um nicht so viel schreiben zu müssen...
\newcommand{\bs}[1]{\boldsymbol{#1}}
\newcommand{\ol}[1]{\overline{#1}}
\newcommand{\wtilde}[1]{\widetilde{#1}}
\newcommand{\mrm}[1]{\mathrm{#1}}
\newcommand{\mbf}[1]{\mathbf{#1}}
\newcommand{\mbb}[1]{\mathbb{#1}}
\newcommand{\mcal}[1]{\mathcal{#1}}
\newcommand{\mfrak}[1]{\mathfrak{#1}}

%Abkürzungen
\newcommand{\zB}{z.\,B.\ }
\newcommand{\bzw}{b.\,z.\, w.\ }
\newcommand{\Dh}{d.\,h.\ }
\newcommand{\Gl}{Gl.\ }
\newcommand{\Abb}{Abb.\ }
\newcommand{\Tab}{Tab.\ }

\usepackage{braket}
\usepackage{cleveref}
\usepackage{graphicx, subfigure}
\usepackage{rotating}
\usepackage{appendix}


\begin{document}
\LabratoryName{KT.HIP}{Higgs physics with the ATLAS experiment}
\ProtocolAuthor{Eric}{Bertok}{eric.bertok@stud.uni-goettingen.de}
\Assistant{K. Abeling}{}
\ResearchFocus{Nuclear and particle physics (M.phy.404)}
\ConductedOn{19}{04}{2018}
\date{\today}
% eines von beiden
\CopyNotWanted
%\CopyWanted

\pagenumbering{roman}
\maketitle

%\begin{otherlanguage}{english}
%\end{otherlanguage}

\tableofcontents

\clearpage
\pagenumbering{arabic}

\section{Introduction}
\label{sec:introduction}

\section{Theory}
\label{sec:theory}
\subsection{A summary of the Standard Model}
The Standard Model (SM) is the combination of different theories governing
three of the four known fundamental forces (electromagnetic, weak and strong
interactions), as well as all known elementary particles and their interaction. 
It consists of 6 quarks making up the hadrons, 6 leptons, as well as 4 gauge
bosons acting as force carriers for the interaction between the other
particles. The interactions are fixed by the principle of local gauge
invariance. The last puzzle piece - the Higgs boson - has recently been found at
the LHC. It gives mass to the other particles by the Higgs mechanism, a form of
spontaneous symmetry breaking. 
The Standard model has been a great success, having made predictions for new
physics as well as confirming these predictions with great accuracy. An example
of this is the theory of quantum-electro-dynamics, which has been confirmed to
an astounding precision [cite].
Despite the great success, the Standard Model is a rather ad-hoc combination of
different ideas without a unifying underlying theoretical principle. It
features 26 free parameters, namely the fermion masses, the coupling strengths
of the fundamental forces, mixing angles of quarks and neutrinos and parameters
specifying the Higgs mechanism \cite[p.\;500]{thomson}. Therefore, it is
desired that all four forces would unify into a ``Great Unified Theory'' (GUT).
There are other obvious shortcommings of the SM: Gravity is not
part of the SM, meaning that general relativity is not compatible with it,
although being much weaker than the other forces, it can be neglected in
particle experiments.
Furthermore, dark matter is not described by the SM. Furthermore, the
mass of the Higgs boson does not have the right mass order of magnitude at very
high energy scales due to loop corrections, a problem known as the ``hierarchy
problem'' \cite[p.\;505]{thomson}.

\subsection{Beyond the Standard Model}
Several ideas exist for an extension of the SM. Supersymmetry is an attempt to
both unify the electroweak and strong force as well as solving the hierarchy
problem. In supersymmetry, every elementary particle would have a corresponding
super-partner, called a ``sparticle''. So far, no super-partners have been
found \cite{thomson}. Also, there would be the need for 5 different Higgs
bosons. There is also a possibility of extra space-like dimensions that are
hidden from us, which would be a possible explanation for the weakness of
gravity. A prominent theory of quantum gravity - string theory - predicts these
extra dimensions. In an experiment, these extra dimensions could manifest as a
large amount of missing energy \cite{zwiebach2004first}. 
Additionally, neutrino masses need to be explained. A prominent idea is that
neutrinos are majorana particles, namely particles being their own
antiparticles.

\subsection{The Higgs mechanism}
\cite[Ch. 17]{thomson}
The Higgs mechanism is needed in the SM to give rise to massive gauge bosons
and fermions in a locally gauge invariant manner.
The problem can be illustrated with a simplified toy model. Consider a mass
term for a vector boson (e.g. a photon) in the Lagrangian:
\begin{align}
    \mathcal{L}_\text{mass}=\frac{1}{2}m_\gamma A_\mu A^\mu.
    \label{eq:massterm}
\end{align}
Such a term is not invariant under a local gauge transformation
\begin{align}
    \partial_\mu\rightarrow D_\mu&=\partial_\mu+i g A_\mu\\
    A_\mu\rightarrow A_\mu'&=A_\mu-\partial_\mu \chi(x).
    \label{eq:gauge}
\end{align}
One now introduces a complex scalar field $\phi(x)$ with a mexican hat
potential $V(\phi)=\mu^2\phi^2+\lambda\phi^4$, which is shown in
\cref{fig:mexhat}. For $\mu^2<0$, $\lambda>0$, the potential minimum shifts
from the origin at $\phi=0$ to a degenerate ring in the complex plane. The
ground state thus chooses a new vacuum in an arbitrary direction with nonzero
$\phi$, a process called ``spontaneous symmetry breaking''. This ground state
can now be expanded around this minimum, choosing a convenient gauge, to
describe the system's low energy excitations in this new vacuum. This gives
rise to a massless particle called the Goldstone boson, as well a massive scalar
particle, called the Higgs boson. The original lagrangian expressed in this
gauge around this new minimum now has a mass term for the vector bosons without
having broken local gauge invariance.
The standard model Higgs has more subtleties arising from the noncommutativity
of the SU(2) and SU(3) gauge symmetries, but the general principle stays the
same.


\begin{figure}[]
    \centering
    \includegraphics[width=0.5\textwidth]{fig/higgspotential}
    \caption{The mexican hat potential for the scalar Higgs boson. The
    symmetric state at $\phi=0$ is spontaneously broken and a new vacuum state
is chosen in a random direction. For a particular choice of ground state, the
low-energy excitations look like a massive scalar field in addition to a
massless Goldstone field. After a local gauge transformation, a mass term
for the vector bosons becomes visible in the lagrangian \cite{mexhat}.}
    \label{fig:mexhat}
\end{figure}

\subsection{Higgs decay and production}
Apart from maybe neutrinos, whose mass-genertating principle is not yet known,
the Higgs boson couples and thus can decay to all SM particles. However, the
top quark is too massive to be a real decay product of the Higgs boson. Since
the Higgs couples proportionally to the mass of the particle, decays to heavier
particles are favoured. The largest decay channels are to bottom quarks
(57.8\%), W bosons (21.6\%) and tau leptons (6,4\%). Even though photons are
massless, they can be the result of a Higgs decay by means of virtual top quark
loops. The Branching ratio to photons is only 0.2\%.
At the LHC, the two most important Higgs production mechanisms are gluon-gluon
fusion and vector boson fusion \cite[p.\;490f]{thomson}. Although the former
has a significantly higher cross section (10$\times$ higher), vector boson fusion is more
practical, since the virtual top quark loops lead to QCD radiation from the
colour field that makes identifying a Higgs decay more challenging.\\
Since Higgs decays are hard to seperate from the usual multi jet events at the
LHC, decay channels with distinctive final state topologies are favoured. These
include $H\rightarrow\gamma\gamma$ and $H\rightarrow ZZ^*\rightarrow
l^+l^-l^+l^-$. In this lab, the four lepton states are the most important. 
These have to be seperated from other four lepton state processes such as
$t\bar t$ fusion, where each of the two top quarks sends out two W bosons, changing
flavour each time. These W's can then each decay into a lepton neutrino pair.
Another background four-lepton process is the combination of a Z decaying into
two leptons together with a $b\bar b$ pair, each decaying into a lepton,
neutrino pair via a W boson.




\section{Experimental setup and methods}
\label{sec:setup}
\subsection{ATLAS detector}
\begin{figure}[ ]
    \centering
    \includegraphics[width=0.7\textwidth]{fig/atlas}
\caption{Cross section of the ATLAS detector \cite{atlas}}
    \label{fig:atlas}
\end{figure}
The ATLAS detector is one of four major experiments at the Large Hadron
Collider at Cern. It is build for general purpose experiments with
propton-proton collisions. Its main goal has been the detection of the Higgs
boson, tests of the Standard Model and search for new physics beyond the SM,
such as supersymmetry and extra dimensions.
The detector itself consists of a inner detector, a calorimeter and the muon
spectrometer. The inner detector is a collection of different systems like a
pixel detector and semiconductor tracker for measuring the direction, momentum
and charge of charged particles, which are brought on a circular path by large
magnetic fields parallel to the beam axis. The transition radiation tracker
additionally provides information on which type of particle was detected. Next,
the electromagnetic and hadronic calorimeters are designed to absorb as much
energy of the produced particles as possble. This yields the direction and
energy of particles as well as a means to distinguish leptons from hadrons. 
Muons however, being a weakly ionising particle mostly flies through these
calorimeters. Therefore, the largest part of the detector is the muon
spectrometer, which is there to give very precise measurements of the muon
momentum.\\
The parameters that are used to describe particle trajectories and
energy-momentum are
\begin{itemize}
    \item The azimuthal angle $\phi$
    \item The pseudorapidity $\eta=-\ln\left[ \tan \left( \frac{\theta}{2}
        \right) \right]$
    \item the $z$ coordinate along the beam axis
    \item the transverse momentum $p_T=\sqrt{p_x^2+p_y^2}$
        
\end{itemize}

\subsection{trigger?}

\subsection{ATLAS event display}
ATLANTIS, the ATLAS event desplay is used to depict the detector responses for
various decay processes graphically to get a feel for the different decay
topologies. From a view along the z-axis aswell as from a side view, the tracks
in the inner detector and the muon spectrometer, as well as the showers in both
calorimeters can be seen. Additional information on the measured quantities can
be obtained upon selection of these features.
\begin{figure}[]
    \centering
        \includegraphics[width=\textwidth]{fig/Zee}
    \caption{Z$\rightarrow e^+e^-$ decay viewed in ATLANTIS}
    \label{fig:atlantis}
\end{figure}


\subsection{Signifikanz????}
\cite{signifikanz}


\section{Analysis}
\label{sec:analysis}
\subsection{Detector Responses to specific processes}
For this part of the lab, simulated data for single particle detection are
evaluated using ATLANTIS. Of course, such a single particle process is highly
unphysical because of background and additional decay products.
\subsubsection{Electron, $e^-$}
For a single electron, a well-defined single track in the inner detector is
observed. Its curvature is due to the electric charge and the magnets which
force the particle on a curved trajectory. Aligning with the track is a small
deposit in the electronagnetic calorimeter right behind the inner detector.
There, electrons radiate photons that lead to electron-positron pair
production. This leads to a shower.\\
In rare cases, the tracks in the inner detector are not consistent with the
showers in the calorimeter. Therefore it is clear that reconstruction errors
are still possible in such simple cases. There is also a possibility that an
electron emits a photon before reaching the calorimeter.

\subsubsection{Muon,$ \mu^-$}
Muons are especially easy to identify by a long track in the muon spectrometer
that is consistent with a track in the inner detector. As they have a mass of
around 100\;GeV, they are minimally ionising, meaning that they pass through
both calorimeters with virtually no trace. This is the reason for the large
muon spectrometers in the first place.

\subsubsection{Photon, $\gamma$}
Photons leave mostly no tracks inside the spectrometers since they are electrically
neutral and don't typically ionize. However, there is the possibility of
electromagnetic radiation leaving ionization tracks in the inner detector which
leads to a possible misidentification as electrons. The calorimeter responses
are equivalent to those of an electron.

\subsubsection{Tau, $\tau$}
Proper identification of tau leptons is very tricky as it is not directly
observable owing to its very short lifetime. Only its decay products are
visible in the detector. It decays via the weak interaction, always producing
at least one tau-neutrino in addition to mostly either an electron-neutrino or
muon-neutrino pair, or a pion. The latter subsequently decay
mostly into $\mu\nu_\mu$ pairs in the case for charge pions or two photons in
the case of neutral pions.
Reconstructing taus is therefore very challenging, due to the plethora of
possible decay products, as well as the neutrinos, which are invisible to the
detector and have to be inferred from indirect measurements of the missing
transverse momentum.\\
The observed tau events are often similar to either the electron or muon case.
But also different topologies are possible, e.g. three tracks and small
deposits in the hadronic calorimeter in addition to deposits in the
electromagnetic calorimeter. Rarely, a muon track is observed in the muon
spectrometer owing to the decay of a neutral pion.

\subsubsection{Dijets}
The dijet events consist of a large amount of tracks and energy deposits.
The strong force does not allow for the free propagation of single quarks or
gluons. Instead, quarks and gluons hadronise creating additional quark pairs
and gluons. This results in cone shaped tracks and deposits in the
calorimeters. These hadron showers -also referred to as jets- are typical in
proton proton collisions and form a large part of the background. Typically two
jets will stand out from the rest in their high energy, leaving the biggest
deposit in the calorimeters. Aditionally, they are minimally
curved in the inner detector due to their large momentum. These two jets are
therefore identified as coming directly from the interactions of the partons
making up the original protons that collided. The rest of the jets originate
from low energy charged particles.

\subsection{Analysis of muon momenutm loss}
\begin{table}
  \centering
  \begin{tabular}{|c|c|c|c|c|c|c|}
  \hline
  Event&\multicolumn{2}{c|}{Inner Detector}&\multicolumn{2}{c|}{Muon Spectrometer}&\multicolumn{2}{c|}{Loss}\\
  &$p_{ID}$ [GeV]&$p_{T,ID}$ [GeV]&$p_{MS}$ [GeV]&$p_{T,MS}$ [GeV]&$p_{loss}$ [GeV]&$p_{T,loss}$ [GeV]\\
  \hline
  1&$-85.28$&$-38.36\pm 1.016$&$-57.62$&$-25.72$&$27.66$&$12.64\pm 1.016$\\
  2&$43.40$&$33.17\pm 0.797$&$47.53$&$36.17$&$-4.13$&$-3.00\pm 0.797$\\
  3&$-241.37$&$-41.85\pm 3.633$&$-240.72$&$-41.67$&$0.65$&$0.18\pm 3.633$\\
  4&$48.89$&$41.96\pm 0.831$&$-48.47$&$41.26$&$0.42$&$0.70\pm 0.831$\\
  5&$-168.16$&$-53.66\pm 2.054$&$-181.32$&$-62.44$&$-13.6$&$-8.78\pm 2.054$\\
  6&$117.32$&$44.65\pm 1.388$&$100.26$&$38.01$&$17.06$&$6.64\pm 1.388$\\
  7&$-71.94$&$-57.39\pm 1.684$&$-68.66$&$-54.77$&$3.28$&$2.62\pm 1.684$\\
  8&$199.91$&$64.54\pm 2.677$&$203.14$&$66.99$&$-3.23$&$-2.45 \pm 2.677$\\
  9&$-57.84$&$-55.54\pm 1.150$&$-53.71$&$-51.16$&$4.13$&$4.38\pm 1.150$\\
  10&$-100.75$&$-37.79\pm 1.185$&$-97.81$&$-36.40$&$2.94$&$1.39\pm 1.185$\\
  11&$38.26$&$37.59\pm 0.653$&$38.18$&$37.48$&$0.08$&$0.11\pm 0.653$\\
  12&$-105.19$&$-59.88\pm 1.766$&$-112.38$&$-64.27$&$-7.19$&$-4.39\pm 1.766$\\
  13&$236.12$&$47.52\pm 3.324$&$267.31$&$53.73$&$-31.19$&$-6.21\pm 3.324$\\
  14&$-131.69$&$-43.88\pm 1.471$&$-129.21$&$-43.21$&$2.48$&$0.67\pm 1.471$\\
  15&$152.24$&$45.66\pm 1.844$&$161.39$&$48.37$&$-9.15$&$-2.71\pm 1.844$\\
  16&$-35.23$&$-33.65\pm 0.539$&$-35.88$&$-33.23$&$-0.65$&$0.42\pm 0.539$\\
  17&$54.19$&$52.80\pm 1.060$&$53.70$&$52.33$&$0.49$&$0.47\pm 1.060$\\
  18&$-84.75$&$-64.49\pm 2.515$&$-71.79$&$-55.02$&$12.96$&$9.47\pm 2.515$\\
  19&$104.26$&$47.56\pm 1.324$&$111.68$&$51.21$&$-7,42$&$-3.65\pm 1.324$\\
  20&$-184.01$&$-35.52\pm 2.24$&$-174.36$&$-33.96$&$9.65$&$1.56\pm 2.284$\\
  \hline
  \end{tabular}
  \caption{Momenta $p$, and transverse momenta $p_T$ for the first twenty
  events, measured both in the inner detector and the muon spectrometer. The
  difference between these $p_\text{loss}$, ${p_T}_\text{loss}$ is also shown.
  Negative losses indicate a gain in momentum after passing the calorimeters.}
  \label{tab:loss}
\end{table}

For the single muon dataset, both the momentum $p$ and the transverse momentum
$p_T$are
read out using ATLANTIS for the first 20 events. Values are read out from the
inner detector (ID) and the muon spectrometer (MS). An uncertainty $\sigma$ is
also given for both values for the inner detector. Contrary to the multi
purpose inner detector, the uncertainty is assumed to be negligible for the
muon spectrometer, since its size allows for large tracks and therefore great
accuracy. Additionally the muon detector elements are specialized for muons.
The loss of (transverse) momentum $\Delta_{{p}_{(T)}}=|p_{ID}|-|p_{MS}|$ is
calculated. Uncertainty is propagated using the gaussian formula. The result is
shown in \cref{tab:loss}.
Interestingly, for 7 out of the 20 events, the muon spectrometer measured a
higher transverse momentum than the inner detector. This is furthermore not
consistent with the uncertainties from the inner detector. [???]








\section{Discussion}
\label{sec:discussion}


\bibliography{literatur}
\newpage

\begin{appendices}
\end{appendices}

\end{document}

