\NeedsTeXFormat{LaTeX2e}[2005/12/01]
%%    2010/04/06 v1.0 Vorlage Master-Forschungspraktikum Versuchsauswertung
%%    based on the 2009/10/14 v0.1 GAUBM template by Prof Pruschke

\documentclass[twoside,        %% zweiseitiges Layout
               BCOR12mm,       %% Bindekorrektur 12 mm
% please comment out if report is in English
%               english,ngerman, %% Dokumentspr. Deutsch, Alternativspr. Englisch
% please remove comment if report is in English 
               ngerman,english, %% Dokumentspr. Englisch, Alternativspr. Deutsch
               fleqn,headsepline=false,footsepline=false
              ]{Vorlage/MFPREPORT}
\makeatletter
\DeclareOldFontCommand{\rm}{\normalfont\rmfamily}{\mathrm}
\DeclareOldFontCommand{\sf}{\normalfont\sffamily}{\mathsf}
\DeclareOldFontCommand{\tt}{\normalfont\ttfamily}{\mathtt}
\DeclareOldFontCommand{\bf}{\normalfont\bfseries}{\mathbf}
\DeclareOldFontCommand{\it}{\normalfont\itshape}{\mathit}
\DeclareOldFontCommand{\sl}{\normalfont\slshape}{\@nomath\sl}
\DeclareOldFontCommand{\sc}{\normalfont\scshape}{\@nomath\sc}
\makeatother

%% Pakete und Definitionen ausgelagert
\input{Vorlage/packages}
\input{Vorlage/layout}

%% einbinden einiger nützlicher Befehle
\input{Vorlage/shortcuts}

%Zur Formatierung in der Matheumgebung
\renewcommand{\t}{\ensuremath{\rm\tiny}} % Tiefgestellter Text in der Matheumgebung wird schoener mit: $\Phi_{\t{Text}}$
\renewcommand{\d}{\ensuremath{\mathrm{d}}} % Die totale Ableitung ist stets aufrecht zu setzen: \d
\newcommand{\diff}[3][]{\ensuremath{\frac{\d^{#1}#2}{\d#3^{#1}}}} % einfache Ableitung nach x: $\ddx{\Phi}$
\newcommand{\pdiff}[3][]{\ensuremath{\frac{\partial^{#1}#2}{\partial#3^{#1}}}} % wie gesprochen, eine partielle Ableitung: \del
\newcommand{\aeqiv}{\ensuremath{\qquad \Longleftrightarrow \qquad}} % Eine Aequivalenz
\newcommand{\folgt}{\ensuremath{\qquad \Longrightarrow \qquad}} % Ein Folgepfeil mit Abstaenden
\newcommand{\corresponds}{\ensuremath{\mathrel{\widehat{=}}}} % Befehl für "Entspricht"-Zeichen
\newcommand{\mi}[1]{\ensuremath{\mathit{#1}}} % italics für griechische Buchstaben in Matheumgebung

%Um nicht so viel schreiben zu müssen...
\newcommand{\bs}[1]{\boldsymbol{#1}}
\newcommand{\ol}[1]{\overline{#1}}
\newcommand{\wtilde}[1]{\widetilde{#1}}
\newcommand{\mrm}[1]{\mathrm{#1}}
\newcommand{\mbf}[1]{\mathbf{#1}}
\newcommand{\mbb}[1]{\mathbb{#1}}
\newcommand{\mcal}[1]{\mathcal{#1}}
\newcommand{\mfrak}[1]{\mathfrak{#1}}

%Abkürzungen
\newcommand{\zB}{z.\,B.\ }
\newcommand{\bzw}{b.\,z.\, w.\ }
\newcommand{\Dh}{d.\,h.\ }
\newcommand{\Gl}{Gl.\ }
\newcommand{\Abb}{Abb.\ }
\newcommand{\Tab}{Tab.\ }

\usepackage{braket}
\usepackage{cleveref}
\usepackage{graphicx, subfigure}
\usepackage{rotating}
\usepackage{appendix}


\begin{document}
\LabratoryName{KT.HIP}{Higgs physics with the ATLAS experiment}
\ProtocolAuthor{Eric}{Bertok}{eric.bertok@stud.uni-goettingen.de}
\Assistant{K. Abeling}{}
\ResearchFocus{Nuclear and particle physics (M.phy.404)}
\ConductedOn{19}{04}{2018}
\date{\today}
% eines von beiden
\CopyNotWanted
%\CopyWanted

\pagenumbering{roman}
\maketitle

%\begin{otherlanguage}{english}
%\end{otherlanguage}

\tableofcontents

\clearpage
\pagenumbering{arabic}

\section{Introduction}
\label{sec:introduction}
The goal of this experiment is the determination the  of the branching ratio of the $W$ boson
BR($W\rightarrow\mu\nu$). First, $W$ and $Z$ bosons are reconstructed using
data provided by the Tevatron collider at Fermilab. By comparing with monte-carlo
simulations, selection parameters are obtained, which allow for clean cuts for
filtering out background events (jets and cosmic source). The mass and the
transverse mass is then determined for the $Z$ and $W$ boson respectively.
Finally the branching ratio is calculated from the number of selected events,
the trigger efficiencies, as well as the reconstruction efficiencies.


\section{Theory}
\label{sec:theory}

\section{Experimental setup and methods}
\label{sec:setup}

\section{Analysis}
\label{sec:analysis}

\section{Discussion}
\label{sec:discussion}


\bibliography{literatur}
\newpage

\begin{appendices}
\section{$Z$ boson additional plots}
\label{app:z}
\begin{sidewaysfigure}
%\begin{figure}[ht!]
     \begin{center}
         \subfigure[]{
            \includegraphics[width=0.45\textwidth]{fig/zmc_cut_muonpt_1.pdf}
        }
        \subfigure[]{
            \includegraphics[width=0.45\textwidth]{fig/data_cut_muonpt_1.pdf}
        }\\ %  ------- End of the first row ----------------------%
        \subfigure[]{
            \includegraphics[width=0.45\textwidth]{fig/zmc_cut_muonpt_2.pdf}
        }
        \subfigure[]{
            \includegraphics[width=0.45\textwidth]{fig/data_cut_muonpt_2.pdf}
        }
    \end{center}
    \caption{
        $p_{T}$ for accepted (black) and rejected (red) events
            for (a,c): the $Z$ monte-carlo data (b,d): the real data.}{
     }
   \label{fig:pt}
%\end{figure}\bibliography{literatur}
\end{sidewaysfigure}
\begin{sidewaysfigure}
%\begin{figure}[ht!]
     \begin{center}
         \subfigure[]{
            \includegraphics[width=0.45\textwidth]{fig/zmc_cut_chisqdof1.pdf}
        }
        \subfigure[]{
            \includegraphics[width=0.45\textwidth]{fig/data_cut_chisqdof1.pdf}
        }\\ %  ------- End of the first row ----------------------%
        \subfigure[]{
            \includegraphics[width=0.45\textwidth]{fig/zmc_cut_chisqdof2.pdf}
        }
        \subfigure[]{
            \includegraphics[width=0.45\textwidth]{fig/data_cut_chisqdof2.pdf}
        }
    \end{center}
    \caption{
        $\chi^2$ for accepted (black) and rejected (red) events
            for (a,c): the $Z$ monte-carlo data (b,d): the real data.}{
     }
   \label{fig:chi}
%\end{figure}\bibliography{literatur}
\end{sidewaysfigure}
\begin{sidewaysfigure}
%\begin{figure}[ht!]
     \begin{center}
        \subfigure[]{
            \label{fig:first}
            \includegraphics[width=0.45\textwidth]{fig/zmc_cut_ehalo1.pdf}
        }
        \subfigure[]{
           \label{fig:second}
            \includegraphics[width=0.45\textwidth]{fig/data_cut_ehalo1.pdf}
        }\\ %  ------- End of the first row ----------------------%
        \subfigure[]{
            \label{fig:third}
            \includegraphics[width=0.45\textwidth]{fig/zmc_cut_ehalo2.pdf}
        }
        \subfigure[]{
            \label{fig:fourth}
            \includegraphics[width=0.45\textwidth]{fig/data_cut_ehalo2.pdf}
        }
    \end{center}
    \caption{
    $E_{\text{halo}}$ for accepted (black) and rejected (red) events
            for (a,c): the $Z$ monte-carlo data (b,d): the real data.}{
     }
  \label{fig:ehalo}

%{()}

%\end{figure}
\end{sidewaysfigure}
\end{appendices}

\end{document}

